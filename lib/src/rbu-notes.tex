
Also note there 2 branches: master and testbranch, Latest is in testbranch (FITS_IF?).
________________________________________________________________________________
Notes on filenames in experimentation:

- FITS*.ad? are vanilla, used in ./examples/* and ./fits/* 
            --> solved parsing arrays by allocating max array NAXS 1..999 (wastes memory)
            --> recognizes only IMAGE_HDU for size calculation and data access
            --> based on Stream_IO, full data access implemented (only IMAGE_HDU)

- FITS_IO is experimental (has some apps in ./examples/ dir)
            --> parsing rewritten: parse by array of card-keys, and fill-in card-values.
            For array keys, allocate only as long NAXIS array as was in the file (needed some tricks)
            --> recognizes all HDU types (for size calculation, not data access)
            --> based on Stream_IO, only Set_index() implemented

- FITS_IF is attempt to define interface
            --> independent of Stream_IO: uses generic with abstract Source_Type and NextBlock() func

TODO clenup and merge the three code bases:

* FITS - internal library with FITS-functionality implementation.
         It is independent of any File_IO (uses Source_Type & NextBlock() generic),
         contains only cumputations/algorithms as described in FITS-standard.
         Its interface should support FITS_IO and some form of 
         NCube-based object-oriented interface (serialization for Ada.Streams).

* FITS_IO - external interface similar to Stream_IO, Direct_IO etc... using FITS-lib.

* FITS_NCube - Ada.Streams based OO-interface (T'Write T'Read)
         * Creation of an FITS NCube object should construct an object in memory,
         which -  by buffering - can seamlessly access all NCube inside the FITS-file
         even if files is "huge".
         * Alternatively consider only providing generic Get_ScanLine Put_ScanLine 
	(as T'Write, T'Read) which the caller can plug-in into his own defined NCube.
         (XY-func use as MyNCube'Write)


_________________________________________________________________________________
Notes on extensibility:

extendeable by:

[1] 
New type besides UInt8 Int16/32/64 Float32/64, for instance Int128
Solution: do generic package parametrized by type (one of above) - e.g.
implementation has no mention of concrete type (only BITPIX checked against
instantiated type) Possible ?? How about Endianness and others?
Example: separate out the 6 FITS types to FITS.Types.ads as INFORMATIVE package
Nothing in FITSlib must depend on FITS.Types, it only list the types currently
in FITS standard. User can always add new type.
So user can instantiate:
package FITS_IO_UInt8 is new FITS_IO(DataType => FITS.Types.UInt8);

There are parts which operate on Header which is fully defined (no need for generics)
Should these be separated ?
FITS.(File.?)Header -> normal package  : Read_Header_Blocks() (file access) 
                                         is base for Set_Index() & Get() 
FITS.(File.?)Data   -> generic package : Read/Write data access file
how related to FITS.File e.g. the file access

[2] 
New card-group for parsing, like: 
* standard conformance [SIMPLE]
* positioning [BITPIX NAXISn]
* scaling [BZERO BSCALE]
* WCS keys (CTYPE CREFn CPIXELn CUNITn...)
etc...

Extendability of HDU/card-groups means:
* add new keywords to existing
* add new parsing code which converts values to structs for 
each new keyword or keyword-array 
(1D array like NAXISn, 
2D arrays/matrices like n WCS keys: CDi_j, PSj_m etc...)

[3] 
New XTENSION-HDU type besides current (IMAGE, ASCII- and BIN-TABLE)
Note: Primary-HDU is fixed and defined, not extendable??

Extendeability of new HDU-type means: 
* define new/additional keywords and 
* struct for holding parsed values.

[3] depends on [2]: new HDU-type has fixed(?) set of keywords,
e.g. must be parseable as [2] requires.
On the other hand, there are keyword groups which are independent
of HDU-type, or are optional, user-defined, etc add-on to keywords
like WCS-coord keys
Sounds like [2] is more general. [3] should build on [2].

[4] What else ?

___________________________
IMPLEMENTATION constructs:
A. procedural by keywords and arrays
B. generics and sub-packages
C. tagged records/classes

Use above three to implement

 Positive Size_blocks(HDU_Size_Type);

needed in Set_Index(File,HDUNum) and
HDUInfo Get(File) and others...

All three will result in more scans of Header from the disk/file.
Parsing certain groups of keys might be needed on different
phases of an application, might be conditional during run time, etc...

Probably impossible to expect parse everything _needed_ with one scan
of Header on disk/file.

How about the way around approach: 
one scan, and parse everythig what is in Header (and known to 
the lib implementation), whether needed or not ?
E.g. convert Header on disk -> records in mem. As much records 
as is implemented in the lib. How about user-extendeabiity then ?


