
-- Interface func description:
--
-- this library supports RANDOM access (Set_Offset) in reading and
-- SEQUENTIAL writing of FITS files
--
-- FIXME?? random access (Set_Offset) not implemented --> should it be here?
--         or only in NCube ?
--
-- Note:
-- Reading Header has 2 variants:
-- 1 if Header is considered "small" -> can read/write one object of
--   Header_Type -> such func (Header'Read/'Write attrib) should
--   handle padding
-- 2 if Header considered "big" -> user has to read/write it in cycle by
--   CardBlocks (Card_Arr'Write/'Read which is defined by language).
--   CardBlock has 36 cards, so it covers the padding.
--
-- Set_Index(in Stream, in HDUNum)
-- position file pointer to begining of Header of the given HDU
--
-- HDU info struct: xtension name, cards cnt, data type, dimensions per axis
-- Get(in Stream, in HDUNum, out HDU_Info_Record)
--
--
-- Read (user knows which coords' data he wants - but does not know
--       where in file it is - and he has not the data)
--
-- Reading is RANDOM ACCESS, user may position to given HDU (Set_Index),
-- and then specify an OFFSET (Set_Offset). Offset is counted from the
-- begining of the Header or DataUnit.
--
-- Set_Offset(Stream, CardIndex)          <- move in Stream to Card given by CardIndex
-- Set_Offset(Stream, CharIndexVector[2]) <- move in Stream to Character given by 2-dimensional CharIndexVector
-- Set_Offset(Stream, ElementIndexVector[N], ElementType)<- move in Stream to Element given by ElementIndex
--
-- Read_Card (in Stream, out Card)
-- Read_Cards (in Stream, out CardBlock)
--
-- 6x Read_Data (in Stream, in (N1,N2,...), out DataArr[<>])
-- -- DataArr is 1-dimensional and has 6 variants: UInt8 Int16/32/64, Float32/64 and is of size N1*N2*...
-- -- (N1,N2,...) is vector which specifies the size of the
-- --  data cube to be read
-- -- It is responsibility of the caller to ensure that we do not
-- -- read behind data unit end (Set_Offset + (N1*N2*...) < DUEnd).
--
-- FIXME we should hide DataArr[] implementation ??
--  future yes: introduce class NCube which will couple Coord and DataArr[]
--
-- 6x function Element(DataArr[<>],(I1,I2,...)) returns Element_Type
--
--
-- Write (user has the data and knows its coordinates
--        but does not know where to write it)
--
-- Writing is alway SEQUENTIAL. So below function need to be called in a loop
-- until all data is written. Useful in scenarios where each data element
-- does not depend on data item in other coordinates, e.g. coordinates are
-- irrelevant: New_Data_Item = func(Old_Data_Item), like scaling,
-- data type change (Int32->Float32), new = log(old), etc...
--
-- Write_Cards(in Stream, in  CardArr[N])
--
-- Write_Padding(in Stream, offset, PadValue)

-- generic
--  ElementType : UInt8 INT16/32/64 Float32/64
--  function Element(Coord : Vector[Dims]) return ElementType;
-- Write_Data(in Stream);
--
-- -- user has to supply the implementation of the Element() function
-- -- Element() function return one data element in coordinats given
-- -- in vector Vector[Dims]
--
--
--
-- Misc
--
-- function Card(key: in String,value: in String,comment=null: in String) return Card_Type
--

-- Extensibility:
-- new groups of keywords in Header (like WCS coords related is now standardized)
-- new Extension HDU types (now: IMAGE, TABLEBIN/ASCII)
-- new Data Unit data-type? (now: UInt8, Int16/32/63, Float32/64)

-- Note: using Read/Write_Card instead of 'Read 'Write attribs
-- does not force use of Stream_IO in client sw (one could implement
-- Read_Card also on Direct_IO Text_IO or or other)
-- However we have SIO.File_Type we should do FITS.File_Type and
-- package FITS renames Ada.Streams.Stream_IO; this allows with
-- one line change to do:
-- package FITS renames Ada.Dierect_IO; and new body with Direct_IO
-- implementation of Read/Write_Card and others...
-- OR even better (rename only File_Type not all Package):
-- type File_Type is new SIO.File_Type;
-- then we'd have: FITS.File.File_Type

-- Note on Read_Data() API usage:
-- This access enables data modification of each data element separately.
-- If modified data element depends on more (maybe neighbpuring) data,
-- this simple interface is cumbersome. Better to use NCube, where access
-- is governed by coordinates.


with Ada.Streams.Stream_IO;

with Ada.Strings.Bounded; use Ada.Strings.Bounded; -- Max20 only FIXME !!

with Keyword_Record; use Keyword_Record; -- FPositive needed
with Strict; -- Positive_Arr needed
with Optional; -- Bounded_String_8 Card_Arr needed 
use Optional; -- Card_Arr used elsewhere then Optional
with FITS; use FITS;

package File is

   package SIO renames Ada.Streams.Stream_IO;

   -- FIXME needed also by Misc
   BlockSize_bits : constant FPositive := 2880 * Byte'Size; -- 23040 bits
    -- [FITS 3.1 Overall file structure]

   CardsCntInBlock : constant Positive := 36;
   type Card_Block is array (Positive range 1..CardsCntInBlock) of Card_Type;
--   type Card_Arr   is array (Positive range <>)                 of Card_Type;
--   for Card_Arr'Size use Card_Arr'Length*(CardSize);
-- FIXME how to guarantee these Arrs are packed OR do we need to guarantee ?
   pragma Pack (Card_Block); -- not guaranteed ??
-- pragma Pack (Card_Arr);   -- FIXME this is only suggestion to compiler



   -----------------------------
   -- Read/Write Header Cards --
   -----------------------------

   -- Read cards one-by-one or by blocks of 36 cards.
   -- ** FITS standard guarantees that a healthy FITS file has at least
   -- ** 36 cards from start.

--   function Read_Card  (FitsFile  : in  SIO.File_Type)
  --   return Card_Type;

--   function Read_Cards (FitsFile  : in  SIO.File_Type)
  --   return Card_Block;

   -- Write cards one-by-one or in block of any number of
   -- cards as suitable to generate them.
   -- ** FITS standard mandates: Header must be
   -- ** closed be END-card and padded (Write_Padding).

--   procedure Write_Card  (FitsFile : in SIO.File_Type;
  --                        Card     : in Card_Type);

 --  procedure Write_Cards (FitsFile : in SIO.File_Type;
   --                       Cards    : in Card_Arr);


   -------------------------------
   -- Write Header/Data Padding --
   -------------------------------

   HeaderPadValue : constant Unsigned_8 := 32; -- Space ASCII value
   DataPadValue   : constant Unsigned_8 :=  0;

   procedure Write_Padding(FitsFile : in SIO.File_Type;
                           From     : in SIO.Positive_Count;
                           PadValue : in Unsigned_8);
   -- [FITS ??]: FITS file consists of 2880-bytes long blocks.
   -- If last Header- or Data-block is not filled up,
   -- Write_Padding puts PadValue from FileOffset until end of the block.
   -- If Block is filled up, Write_Padding does nothing.


   ---------------------
   -- Read/Write Data --
   ---------------------

   -- Read data by free-sized data-chunks into 1-dim array
   -- max size is known from header Get(HDU_Info): NAXIS1 x NAXIS2 x ...
   -- ** FITS standard guarantees that healthy FITS File has that many data.

--   generic
--     type Data_Arr(<>) is private;
--   procedure gen_Read_Data (FitsFile : in  SIO.File_Type;
--                            Data     : in out Data_Arr);
   -- Data_Arr is any of FITS.UInt8_Arr ... FITS.Float64_Arr

--   generic
--     type Data_Arr(<>) is private;
--   procedure gen_Write_Data (FitsFile : in SIO.File_Type;
 --                            Data     : in Data_Arr);
   -- Data_Arr is any of FITS.UInt8_Arr ... FITS.Float64_Arr

   -- FIXME should also above generic's be based on Item (Item_Arr) as below ?
   --       e.g. no need to define arrays as in FITS.ads
   --       instantation can be done really by types rather then array types
   --       Q: How to guarantee then that the Item_Arr array would be packed?
   --       A: At instation the supplied array must be packed.
   --          No represantation clauses allowed in generic
   --
   generic
    type Item is private;
--    type Item_Arr is array (FPositive range <>) of Item;
    with function Element (Coord : in Strict.Positive_Arr) return Item;
   procedure Write_DataUnit (FitsFile  : in  SIO.File_Type;
                             MaxCoords : in  Strict.Positive_Arr);
   -- Item is any of FITS.UInt8 ... FITS.Float64
   -- Notes:
   -- Write_Data writes the complete DataUnit
   -- Write_Data adds DataUnit padding after last Item written


   -----------------------
   -- FITS file content --
   -----------------------
   package Max20 is
        new Ada.Strings.Bounded.Generic_Bounded_Length (Max => 20);

   type HDU_Info_Type(NAXIS : Positive) is record
      XTENSION : Max20.Bounded_String;   -- XTENSION string or empty
      CardsCnt : FPositive;       -- number of cards in this Header
      BITPIX   : Integer;             -- data type
      NAXISn   : Strict.Positive_Arr(1..NAXIS); -- data dimensions
   end record;

   function  Get (FitsFile : in  SIO.File_Type)
      return HDU_Info_Type;


   -- read optional cards given by Keys
   -- FIXME consider renaming Get* to Read_* because they move File_Index

   function BS8(S : String; Drop :Ada.Strings.Truncation := Ada.Strings.Error) 
	return Bounded_String_8.Bounded_String
		renames Bounded_String_8.To_Bounded_String;

   Descriptive_Keys : constant Optional.Bounded_String_8_Arr :=
	(BS8("DATE"),BS8("REFERENC"),BS8("ORIGIN"),BS8("EXTEND"),BS8("BLOCKED"));

   Observation_Keys : constant Optional.Bounded_String_8_Arr :=
	(BS8("DATE-OBS"),BS8("DATExxxx"),
	 BS8("TELESCOP"),BS8("INSTRUME"),BS8("OBSERVER"),BS8("OBJECT"));
	-- FIXME DATExxxx needs special handling!?

   -- NOTE bounded string operator * repeats the string (and performs the conversions??)
   -- so it can be (mis)used to initialize bounded_string with string like this:
   use Bounded_String_8;
   Biblio_Keys : constant Optional.Bounded_String_8_Arr := 
		(1 * "AUTHOR", 1 * "REFERENC");

   Commentary_Keys : constant Optional.Bounded_String_8_Arr := 
		(1 * "COMMENT", 1 * "HISTORY", 8 * " ");

   Array_Keys : constant Optional.Bounded_String_8_Arr := 
		(1 * "BSCALE", 1 * "BZERO", 1 * "BUNIT", 1 * "BLANK",
		 1 * "DATAMAX", 1 * "DATAMIN");

-- WCS keys : ... see Section 8

   Extension_Keys : constant Optional.Bounded_String_8_Arr :=
	(1 * "EXTNAME", 1 * "EXTVER", 1 * "EXTLEVEL");

   Reserved_Keys : constant Optional.Bounded_String_8_Arr :=
	(Descriptive_Keys & Observation_Keys & Biblio_Keys & Array_Keys);

   function  Get_Cards (FitsFile : in  SIO.File_Type;
			Keys : in Optional.Bounded_String_8_Arr)
      return Card_Arr;
-- FIXME consider returning also position at which the card was in the Header
-- e.g. some array of records needed -> then consider returning also 
-- separately key and value/comment??

   -------------------------
   -- Positioning in file --
   -------------------------

   procedure Set_Index(File : in SIO.File_Type;
                       HDUNum   : in Positive);





   -- Misc FIXME

	function  DU_Size_blocks (FitsFile : in SIO.File_Type) return Positive;	
	-- used in commands
private
	-- for Misc subpackage
	
	function  HDU_Size_blocks (FitsFile : in SIO.File_Type) return Positive;	

end File;
