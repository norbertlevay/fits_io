__________________________________________________
Two issues which lead to Numeric_Type introduction

[1] 
the solution with Tf Tc Tm as generic params for type in Fits-file, type in which scaling is calculated (always float) and type in which caller receives the data, respectively ended up
with excessive amount of with-ed func parameters. These needed to be repeated in applications
as well because each package re-intruduced these types again, however

pack1.Tf /= pack2.Tf

so all with-func needed to be repeated. Issue is somewhat remedied if Child packages are derived
fomr first packeg which has all with-s, thus sharing the types Tf Tc Tm, then

pack1.Tf.child1(Tf) = pack1.Tf.child(Tf)

both children see the same type. --> Collect all type related functionality (conversions) into
one package and root all other packages on that.


[2]
type-symmetry and Read/Write : don't convert Tc->Tf but always go 2 steps: Tf->Tc and Tc->Tm
then code can be shared between read-write swapping the types.
E.g. rather then trying to convert between all possible combinations of Tf-Tm,
divide the Linear func at Float (it becomes type symmetric)):

Vout = conv( A + B * conv(Vin) )
            ^^^^^^^^<- this part is in Float: divide Raw and Physical here:

E.g.
ops in Type1 ...
Fin = conv(Vin)
Fout = A + B * Fin  <--- central part in float symmetric operation before and after
Vout = conv(Fout)
... ops in Type2

(see git-tag 'numeric-with-undefs-and-scaling-linear-symmetric')
Raw outputs Float(Tc) and Physical accepts Flost(Tc) in Read direction
Physical has data and converts them to Float and Raw accepts Float (converts to Tf) and writes.

Introduce symmetric handling: Tf->Tc and separately Tc->Tm
then swapping Tm->Tc (needed in Write) leads to the same code as in Tf->Tc

Tf->Tc and Tm->Tc e.g. only Tx->Tc 

Read  direction: Tx(=Tf)->Tc and Tc->Tx(=Tm)
Write direction: Tx(=Tm)->Tc and Tc->Tc(=Tf)

where Tc is some Float-type (of suitable resolution with respect to Tm Tf).







__________________________________________________________________
Handling of Undefined values is "surprisingly complicated" because

* for Float undefined values is NaN (which is out-of-range of valid values, good!) 

* for Integers undefined value is reserved _from_ range of possible values

* for (Unsigned)Integers undefined value is optional (if BLANK present in Header (Read),
or for Write: user knows his data: might have data with or without undefined values

* for Floats we should always assume possible NaN (e.g. undefined values presetn) ??

* also user _may_ supply undefined value in target type (for instance
source-type (BLANK) it was 1 but in target user would like to place to T'Last)
in this case it should take precedence over UndefOut = A + B * UndefIn

* and incase converting from Float-> Integer user _must_ supply Undef
because it caanot ba calculated from Tsrc Float.NaN

* because user may/must specify target-undefined value, and that may be
incorrect: being in-range of target-valid values, we must check
that Vout=UndefOut only-and-only-if Vin=UndefIn




__________________
Renaming a type

"A subtype of an indefinite subtype that does not add a constraint only
introduces a new name for the original subtype (a kind of renaming under a different notion).

subtype My_String is String;

My_String and String are interchangeable."

From: https://en.wikibooks.org/wiki/Ada_Programming/Type_System#Anonymous_subtype





