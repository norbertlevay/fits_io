
- NOTE it is the caller who determines conversions by choosing Tin Tout
- -> all combinations should be permissible and should result in minimal loss of accuracy
-    for the given case (choice of scaling domain)
--- Library must provide all patches for possible type combinations:
 generics instantations should not fail!!!

Conversion between all types:
- if out of range (no clipping) but Constraint_Error raised (like Float(8989) -> UInt8)
- if in range, in case of floats only precision is of concern (number of digits lost)
---> NOTE that all conbinations permissible means data values may be out of 
	range (Constrant_Error will be raised) and/or may suffer loss of precision 
	It is the responsability of the user to know  which conversions 
	make sense for his data


- NOTE Tab11 vals is not conversion but two interpretaions of the same bit pattern
- once as sign number other as unsigned number
--> FIXME How to recognize when to do SignConversion by flippinMSB ?

- NOTE Scaling:  
    +(BZERO + BSCALE * (+Vin)  calc in which domain ?
OR    BZERO + BSCALE * (+Vin)  calc in Tout domain (note no result-converion)




BZERO BSCALE defaults to 0.0 1.0 if not given in Header -> e.g. ALLWAYS available
				
Integers:			BLANK	BZERO BSCALE		func name
raw => flipSign		     flipBLANK	Tab11			SignConverted_Value
raw => checkI -> conv		  y	0.0 1.0  		opt Checked_TypeConverted_Value
raw => conv		  	  n	0.0 1.0  		opt TypeConverted_Value
raw => checkI -> conv -> scale	  y 	none of above		Checked_Scaled_Value
raw => conv -> scale		  n	none of above		Scaled_Value


Floats:
raw => checkF			  -	0.0 1.0			Checked_Value
raw => checkF -> conv -> scale	  -	not the above		Checked_Scaled_Value



Header analyzes:

BITPIX		BLANK	BZERO BSCALE		output
 > 0		 any	Tab11			SIGN_CONV (convert also BLANK -> new BLANK)
 > 0		  y	0.0 1.0			CHECKED_TYPE_CONV
 > 0		  n	0.0 1.0 		TYPE_CONV
 > 0		  y	otherwise		CHECKED_SCALING
 > 0		  y	otherwise		SCALING

 < 0		  -	0.0 1.0 or not given	CHECKED_FLOAT	
 < 0		  -	given			CHECKED_FLOAT_SCALING	

