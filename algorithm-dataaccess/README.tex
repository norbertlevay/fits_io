


Implements:

- Endianness for all types
- Data Block definition, generic and so extendable to new types (like 128 bit Integer/Float)
- 'Read 'Write Block attribs (2880bytes) of any BITPIX type
- DataBlock position calculations in HDU and File
- Physical-Array value conversion: for BITPIX-Integer types: scaling values from Array-values to Physical-values

___________________________________________
Physical <-> Array value conversion:

if one of BZERO,BSCALE present in Header other not, use default for the missing key

BITPIX   	BZERO 	BSCALE		Note
 >0 (Int)	none	none		implement nothing, user uses raw values directly
 >0 (Int)	Tab11	1.0		phys is Integer as given in Tab11; implement Unsigned - Signed conversion : This is special case of the next 2! No extra implementation?
 >0 (xInt8..32)	yes	yes		phys in Float32	by Eq(3) <-- FIXME as below: how does accuracy of Int32 relate to Float32?
 >0 (Int64)	yes	yes		phys in Float64	by Eq(3) <-- FIXME check! is this correct? INt64 really implies Float64?? probably not

 <0 (FloatXX)	none	none		implement nothing, user use raw
 <0 (Float32)	yes	yes		phys is Float32 use Eq(3)
 <0 (Float64)	yes	yes		phys is Float64 use Eq(3)

E.g. implement:
1, PhysVal type Integer: implement for all integer types Signed-Unsigned conversions (flip highest bit)

2, PhysVal type Float:
2a, for all integer-float combinations: Eq(3) as generic: BSCALE BZERO-> Float types,  BLANK & Data -> Integer types
2b, for     float  -float combinations: Eq(3) as generic: BSCALE BZERO-> Float types,  Data -> Float types
